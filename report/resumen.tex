\pagestyle{empty}

\begin{center}
    \textbf{Resumen}
\end{center}

La rotonda del Padre Anchieta (Tenerife) está situada en una posición estratégica al conectar con el corazón académico, de ocio, comercial y urbanístico de la ciudad de La Laguna; por lo que asume una cantidad sustancial de tráfico que, en horas punta, suele causar atascos importantes. Con el objetivo de aliviar la congestión del tráfico, este proyecto estudia la instalación de semáforos en la rotonda, optimizando la duración de las fases de estos gracias a un algoritmo evolutivo. Para ello, emplea SUMO para llevar a cabo la simulación del tráfico.

\medskip

El proyecto emplea un mapa de la zona obtenido de OpenStretMap, y el flujo de tráfico es generado gracias a datos del Cabildo de Tenerife. Se ha llevado a cabo un estudio estadístico previo para determinar los mejores parámetros del algoritmo evolutivo. Se concluyó que los resultados de la simulación indicaban que el empleo de los semáforos no consigue mejorar el tráfico de la rotonda.

\medskip

\noindent \textbf{Palabras clave:} SUMO, simulación, tráfico, algoritmos evolutivos, TypeScript, semáforos, planificación, optimización

\newpage
\pagestyle{empty}

\bigskip
\bigskip

\begin{center}
    \textbf{Abstract}
\end{center}

The Padre Anchieta roundabout (Tenerife) is strategically located as it connects the academic, leisure, commercial and urban heart of the city of La Laguna; therefore it assumes a substantial amount of traffic which, at peak times, often causes major traffic jams. With the aim of relieving traffic congestion, this project studies the installation of traffic lights in the roundabout, optimizing the duration of the phases of these thanks to an evolutionary algorithm. To this end, it uses SUMO to carry out traffic simulation.

\medskip

The project uses a map of the area obtained from OpenStretMap, and the traffic flow is generated thanks to data from the Tenerife Island Council. A previous statistical study has been carried out to determine the best parameters of the evolutionary algorithm. It was concluded that the results of the simulation indicated that the use of the traffic lights does not manage to improve the traffic of the roundabout.

\medskip

\noindent \textbf{Keywords:} SUMO, traffic, simulation, evolutionary algorithms, TypeScript, traffic lights, scheduling, optimization

